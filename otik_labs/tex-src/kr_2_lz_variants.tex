%%%%%%%%%%%%%%%%%%%%%%%%%%%%%%%%%%%%%%%%%%%%%%%%%%%%%%%%%%%%%%%%%%%%%%%%%%%%%%%
% Список вопросов и билеты по 2 вопроса
% расстояния подобраны методом тыка
%%%%%%%%%%%%%%%%%%%%%%%%%%%%%%%%%%%%%%%%%%%%%%%%%%%%%%%%%%%%%%%%%%%%%%%%%%%%%%%
\documentclass[a4paper,14pt]{extarticle}
\usepackage[
  a4paper, mag=1000, %nohead,includefoot,
  margin=12mm, top=5mm %headsep=20mm, footskip=0mm
]{geometry}

\usepackage[T2A]{fontenc}
\usepackage[utf8x]{inputenc}
\usepackage[english,russian]{babel}
% \usepackage{cyrtimes}	% Times, Arial и Courier одновременно
\usepackage{graphicx}
\usepackage{enumitem}	% Настраиваемые списки
\newcommand\No{№}
\usepackage{mathtools}  % amsmath with extensions
\usepackage{amssymb}  % (otherwise \mathbb does nothing)
\DeclareMathOperator{\sal}{sal}
\DeclareMathOperator{\sar}{sar}
\DeclareMathOperator{\shl}{shl}
\DeclareMathOperator{\shr}{shr}
\newcommand{\hex}[1]{\ensuremath{\mathtt{#1}}}


\usepackage[pdftex,unicode=true,colorlinks]{hyperref}
    \hypersetup{citecolor=blue,linkcolor=blue,urlcolor=blue}
    

\newcounter{count_variants}
\setcounter{count_variants}{1}



% Вариант с 3 семействами; аргументы -- 1=m, 2=RLE, 3=LZ77, 4=LZ78 ============
% \newcommand\vRZZ[4]{
% \begin{center}
% Вариант 4-\arabic{count_variants}\addtocounter{count_variants}{1}
% \end{center}
% 
% $m = \hex{#1}$
% \bigskip
% 
% Размер байта $k=3$ бита. Закодировать $m$ алгоритмами подсемейств.
% \begin{enumerate}
% \if #2\empty \else \item #2. \fi
% \if #3\empty \else \item #3. \fi
% \if #4\empty \else \item #4. \fi
% % ^^^ работает, но Вариант 4- ещё надо менять на 3- или 2-. И нудобны пустые скобки
% \end{enumerate}
% 
% \clearpage
% }

% База; аргументы -- 1=размер команды, 2=\item + коды, 3=«указанными кодами»/«указанными кодом» 
\newcommand\vRZZ[3]{
Авторы:\hfill\the\year-\the\month-«\hspace{2em}»
\bigskip

% \begin{center}
Вариант #1-\arabic{count_variants}\addtocounter{count_variants}{1}.
Закодировать $C$ #3. Размер байта $k=3$ бита. 
% \end{center}
\medskip

$C = \hex{\printnextmsg}$
% \bigskip

\begin{enumerate}
#2
\end{enumerate}

\clearpage
}

% \newcommand\vCCC[3]{\vRZZ{4}{ \item #1. \item #2. \item #3. }{указанными кодами}}
\newcommand\vCC[2]{\vRZZ{2(2)}{ \item #1. \item #2. }{указанными кодами} }
\newcommand\vC[1]{\vRZZ{2(1)}{ \item #1. }{указанным кодом} }


% #2
% 
% \newcommand\vCCC[3]{\vRZZ{4}{\begin{enumerate} \item #1. \item #2. \item #3. \end{enumerate}} }
% \newcommand\vCC[2]{\vRZZ{3}{\begin{enumerate} \item #1. \item #2. \end{enumerate}} }
% \newcommand\vC[1]{\vRZZ{1}{  #1. } }


% =============================================================================

\newcounter{count_msg}	% счётчик m
\setcounter{count_msg}{0}
\newcommand\mkmsgname[1]{q\arabic{#1}}

\makeatletter
% Печать текста по счётчику (1, 2, 3,... count_msg)
\newcommand\printmsg[1]{\expandafter\@nameuse{\mkmsgname{#1}}}

\newcommand\defmsg[1]{%
  \stepcounter{count_msg}%
  \global\expandafter\@namedef{\mkmsgname{count_msg}}{#1}%
}
\makeatother

% =============================================================================
\newcounter{msg_idx}	% счётчик m
\setcounter{msg_idx}{0}

\newcommand\printnextmsg{%
    \addtocounter{msg_idx}{1}%
    \ifnum \value{msg_idx}>\value{count_msg}
        \setcounter{msg_idx}{1}%
    \fi
    \printmsg{msg_idx}
    }
% =============================================================================



% \usepackage[small,indentafter]{titlesec}
% \titlespacing*{\section}{0ex}{3.5ex plus 1ex minus .2ex}{2.3ex plus 0ex}
% \titleformat{\section}{\filright\bfseries}{\thesection.}{.5em}{}
% 
% \usepackage{titletoc}
% \titlecontents{section}[1.5cm]{}{\thecontentslabel.~}{}{\titlerule*[9pt]{.}{\contentspage}}
%%%%%%%%%%%%%%%%%%%%%%%%%%%%%%%%%%%%%%%%%%%%%%%%%%%%%%%%%%%%%%%%%%%%%%%%%%%%%%%
\begin{document}
\pagestyle{empty}

% \defmsg{0123\,0123\,0123\,4567~~7777\,7777\,7700\,0000~~1122\,3344\,5566\,4455}
% \defmsg{0000\,0000\,0123\,4567~~4566\,4566\,4566\,6666~~1122\,3344\,5577\,4455}
% \defmsg{0000\,0123\,0000\,0123~~4567\,4567\,4567\,4566~~1122\,3344\,5577\,7777}

\defmsg{0123~0123~0123~4567~~7777~7777~7700~0000}
\defmsg{0000~0000~0123~4567~~4566~4566~4566~6666}
\defmsg{0000~0123~0000~0123~~4567~4567~4567~4566}
\defmsg{0101~0101~0101~0123~~4567~6767~6767~6767}
\defmsg{0123~4567~4567~4566~~6666~6655~5555~0000}

\defmsg{0123~3333~4567~4567~~4567~7777~7777~7777}
\defmsg{0123~4567~4567~4567~~7777~7555~4444~3333}
\defmsg{0123~4567~0123~4567~~7777~7222~2222~2223}
\defmsg{0001~0001~1110~1110~~0001~1110~0123~4567}
\defmsg{0000~0001~1111~1110~~0123~4567~4567~4555}

\defmsg{0000~0000~0123~4567~~0123~0170~1230~0000}
\defmsg{0000~1111~2222~3333~~4444~5555~5555~5567}
\defmsg{0000~1010~1010~4567~~0123~0122~2222~0122}



\setcounter{count_variants}{1}

\vC{RLE с~флаг-битом, код $L$ с~максимальным смещением}
\vC{LZ77 с~односимвольным префиксом, $|S|=3$ бита, код $L$ и~$S$ с~максимальным смещением}
\vC{RLE с~односимвольным префиксом, код $L$ с~максимальным смещением}
\vC{LZ77 с~флаг-битом ссылка/символ, $|S|=3$ бита, код $L$ и~$S$ с~максимальным смещением}
%\vC{LZ78-концепт}
%\vC{LZW (семейство LZ78)}

\vC{RLE с~флаг-битом, код $L$ со смещением $1$}
\vC{LZ77 с~односимвольным префиксом, $|S|=3$ бита, код $L$ и~$S$ со смещением $1$}
\vC{RLE с~односимвольным префиксом, код $L$ со смещением $1$}
\vC{LZ77 с~флаг-битом ссылка/символ, $|S|=3$ бита, код $L$ и~$S$ со смещением $1$}
%\vC{LZ78-концепт}
%\vC{LZW (семейство LZ78)}


\vC{RLE с~флаг-битом, код $L$ с~максимальным смещением}
\vC{LZ77 с~односимвольным префиксом, $|S|=4$ бита, код $L$ и~$S$ с~максимальным смещением}
\vC{RLE с~односимвольным префиксом, код $L$ с~максимальным смещением}
\vC{LZ77 с~флаг-битом ссылка/символ, $|S|=4$ бита, код $L$ и~$S$ с~максимальным смещением}
%\vC{LZ78-концепт}
%\vC{LZW (семейство LZ78)}

\vC{RLE с~флаг-битом, код $L$ с~максимальным смещением}
\vC{LZ77 с~односимвольным префиксом, $|S|=3$ бита, код $L$ и~$S$ с~максимальным смещением}
\vC{RLE с~односимвольным префиксом, код $L$ с~максимальным смещением}
\vC{LZ77 с~флаг-битом ссылка/символ, $|S|=3$ бита, код $L$ и~$S$ с~максимальным смещением}
\vC{LZ77-концепт, $|S|=3$ бита}
%\vC{LZ78-концепт}
%\vC{LZW (семейство LZ78)}

\vC{RLE с~флаг-битом, код $L$ со смещением $1$}
\vC{LZ77 с~односимвольным префиксом, $|S|=4$ бита, код $L$ и~$S$ со смещением $1$}
\vC{RLE с~односимвольным префиксом, код $L$ со смещением $1$}
\vC{LZ77 с~флаг-битом ссылка/символ, $|S|=4$ бита, код $L$ и~$S$ со смещением $1$}
\vC{LZ77-концепт, $|S|=4$ бита}
%\vC{LZ78-концепт}
%\vC{LZW (семейство LZ78)}



% \setcounter{count_variants}{1}
% 
% \vCCC
% {RLE с~флаг-битом}
% {LZ77 с~флаг-битом ссылка/цепочка, $|S|=3$ бита}
% {LZ78-концепт}
% 
% \vCCC
% {RLE с~односимвольным префиксом}
% {LZ77 с~односимвольным префиксом, $|S|=3$ бита}
% {LZW (семейство LZ78)}
% 
% \vCCC
% {Наивный RLE}
% {LZ77-концепт, $|S|=3$ бита}
% {LZ78-концепт}
% 
% \vCCC
% {RLE с~флаг-битом}
% {LZ77 с~флаг-битом ссылка/символ, $|S|=3$ бита}
% {LZ78-концепт}
% 
% \vCCC
% {RLE с~односимвольным префиксом}
% {LZ77 с~односимвольным префиксом, $|S|=3$ бита}
% {LZW (семейство LZ78)}
% 
% \vCCC
% {Наивный RLE}
% {LZ77-концепт, $|S|=3$ бита}
% {LZ78-концепт}


\setcounter{count_variants}{1}

\vCC
{RLE с~флаг-битом}
{LZ77 с~флаг-битом ссылка/цепочка, $|S|=3$ бита}

\vCC
{RLE с~односимвольным префиксом}
{LZ77 с~односимвольным префиксом, $|S|=4$ бита}

\vCC
{Наивный RLE}
{LZ77-концепт, $|S|=3$ бита}

\vCC
{RLE с~флаг-битом}
{LZ77 с~флаг-битом ссылка/символ, $|S|=3$ бита}

\vCC
{Наивный RLE}
{LZ77-концепт, $|S|=3$ бита}

\vCC
{RLE с~односимвольным префиксом}
{LZ77 с~односимвольным префиксом, $|S|=3$ бита}

\vCC
{Наивный RLE}
{LZ77-концепт, $|S|=3$ бита}

\vCC
{RLE с~флаг-битом}
{LZ77 с~флаг-битом ссылка/символ, $|S|=3$ бита}

\vCC
{RLE с~односимвольным префиксом}
{LZ77 с~односимвольным префиксом, $|S|=3$ бита}

\vCC
{RLE с~флаг-битом}
{LZ77 с~флаг-битом ссылка/символ, $|S|=4$ бита}

% \vCC
% {Наивный RLE}
% {LZ77-концепт, $|S|=4$ бита}


\vCC
{RLE с~односимвольным префиксом}
{LZ77 с~односимвольным префиксом, $|S|=4$ бита}








\end{document}
