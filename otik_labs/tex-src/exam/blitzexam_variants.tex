%%%%%%%%%%%%%%%%%%%%%%%%%%%%%%%%%%%%%%%%%%%%%%%%%%%%%%%%%%%%%%%%%%%%%%%%%%%%%%%
% Список вопросов и билеты по 2 вопроса
% расстояния подобраны методом тыка
%%%%%%%%%%%%%%%%%%%%%%%%%%%%%%%%%%%%%%%%%%%%%%%%%%%%%%%%%%%%%%%%%%%%%%%%%%%%%%%
\documentclass[a4paper,14pt]{extarticle}
\usepackage[
  a4paper, mag=1000, %nohead,includefoot,
  margin=12mm, 
  %headsep=20mm,qsfootskip=0mm
  top=1mm, 
  bottom=1mm 
]{geometry}

\usepackage[T2A]{fontenc}
\usepackage[utf8x]{inputenc}
%\usepackage[english,russian]{babel}
\usepackage[russian]{babel}
% \usepackage{cyrtimes}	% Times, Arial и Courier одновременно
\usepackage{graphicx}
\usepackage{enumitem}	% Настраиваемые списки
\newcommand\No{№}
\usepackage{mathtools}  % amsmath with extensions
\usepackage{amssymb}  % (otherwise \mathbb does nothing)
\DeclareMathOperator{\sal}{sal}
\DeclareMathOperator{\sar}{sar}
\DeclareMathOperator{\shl}{shl}
\DeclareMathOperator{\shr}{shr}
\newcommand{\hex}[1]{\ensuremath{\mathtt{#1}}}

\usepackage{icomma}

\usepackage[pdftex,unicode=true,colorlinks]{hyperref}
    \hypersetup{citecolor=blue,linkcolor=blue,urlcolor=blue}
    
\usepackage{tikz}

\newcounter{count_variants}
\setcounter{count_variants}{1}


% \usepackage[small,indentafter]{titlesec}
% \titlespacing*{\section}{0ex}{3.5ex plus 1ex minus .2ex}{2.3ex plus 0ex}
% \titleformat{\section}{\filright\bfseries}{\thesection.}{.5em}{}
% 
% \usepackage{titletoc}
% \titlecontents{section}[1.5cm]{}{\thecontentslabel.~}{}{\titlerule*[9pt]{.}{\contentspage}}
%%%%%%%%%%%%%%%%%%%%%%%%%%%%%%%%%%%%%%%%%%%%%%%%%%%%%%%%%%%%%%%%%%%%%%%%%%%%%%%
\begin{document}
\pagestyle{empty}



% \newcommand\beforetask{\noindent\textbf{ОТИК-\the\year-\arabic{count_variants}\addtocounter{count_variants}{1}.}~}
% \newcommand\aftertask{\par\bigskip\bigskip\bigskip}
% \newcommand\beforetask{\noindent\textbf{ОТИК-\the\year.~}}
% \newcommand\aftertask{\par\vfill\pagebreak[3]}
\newcommand\beforetask{\pagebreak[3]\noindent}
% \newcommand\aftertask{\par\bigskip\par\vfill\pagebreak[3]}
\newcommand\aftertask{\par\par\vfill\pagebreak[3]}


%\newcommand\qa[1]{{\beforetask}#1{\aftertask}}

\newcommand\qtask[2]{{\beforetask}Студент, выполняя #1, нашёл:\\ 
#2
\\
\mbox{Правдоподобны ли результаты?}
% Если нет "--- в~чём они противоречивы?%
Если нет "--- где противоречие?%
{\aftertask}}

\newcommand\qtaskKrHf[1]{\qtask{Кр1 (блочные коды без учёта контекста) по~ОТИК}{#1}}
\newcommand\qtaskKrBoth[1]{\qtask{Кр1 (блочные коды без учёта контекста) и~Кр2 (коды с~учётом контекста) по~ОТИК}{#1}}

\newcommand\qhf[3]{\qtaskKrHf{$\left\{
\begin{array}{@{~}c@{~=~}l}
|C| & #1 ~\text{бит}, \\
I_{\text{БП}}(C) & #2 ~\text{бит}, \\
|\text{Хф}(C)| & #3 ~\text{бит}. \\
\end{array} 
\right.
$}}

\newcommand\qsf[3]{\qtaskKrHf{$\left\{
\begin{array}{@{~}c@{~=~}l}
|C| & #1 ~\text{бит}, \\
I_{\text{БП}}(C) & #2 ~\text{бит}, \\
|\text{ШФ}(C)| & #3 ~\text{бит}. \\
\end{array} 
\right.
$}}

\newcommand\qhfsf[4]{\qtaskKrHf{$\left\{
\begin{array}{@{~}c@{~=~}l}
|C| & #1 ~\text{бит}, \\
I_{\text{БП}}(C) & #2 ~\text{бит}, \\
|\text{Хф}(C)| & #3 ~\text{бит}, \\
|\text{ШФ}(C)| & #3 ~\text{бит}. \\
\end{array} 
\right.
$}}


\newcommand\bpLZ[4]{\qtaskKrBoth{$\left\{
\begin{array}{@{~}c@{~=~}l}
|C| & #1 ~\text{бит}, \\
I_{\text{БП}}(C) & #2 ~\text{бит}, \\
|\text{#4}(C)| & #3 ~\text{бит}. \\
\end{array} 
\right.
$}}


\newcommand\mmLZ[4]{\qtaskKrBoth{$\left\{
\begin{array}{@{~}c@{~=~}l}
|C| & #1 ~\text{бит}, \\
I_{\text{М1}}(C) & #2 ~\text{бит}, \\
|\text{#4}(C)| & #3 ~\text{бит}. \\
\end{array} 
\right.
$}}


\newcommand\qbmdl[5]{\qtaskKrBoth{$\left\{
\begin{array}{@{~}c@{~=~}l}
|C| & #1 ~\text{бит}, \\
I_{\text{#2}}(C) & #3 ~\text{бит}, \\
|\text{#4}(C)| & #5 ~\text{бит}. \\
\end{array} 
\right.
$}}


\setcounter{count_variants}{1}


% оригинальные из Кр1 2024
% % фикс, Iбп, Xф, ШФ 
% \qsf  {36}{23,47}     {26,964} % длина ШФ — нецелая
% \qhfsf{36}{24,6}{27} {26}    % Хаффман хуже ШФ
% \qhf  {36}{25,5}{26}         % ок
% \qsf  {36}{25,7}     {27}    % ок
% % \qsf  {36}{25,7}     {27}    % ок
% \qsf  {36}{26,2}     {28}    % ок
% \qsf  {36}{26,7}     {27}    % ок
% \qsf  {36}{26,8}     {27}    % ок
% \qhfsf{36}{27,5}{28}{28}     % ок
% \qhf  {36}{28,0}{40}         % Хаффман неоптимален и даже хуже фиксированного
% \qsf  {36}{28,2}     {29}    % ок
% \qhf  {36}{28,4}{29}         % ок
% \qhfsf{36}{28,4}{29}{30}     % ок
% \qhf  {36}{28,7}{30}         % ок
% \qsf  {36}{29,0}     {30}    % ок
% % \qsf  {36}{29,0}     {30}    % ок, дубль
% \qsf  {36}{29,1}     {30}    % ок
% \qsf  {36}{29,2}     {30}    % ок
% \qhf  {36}{29,5}{30}         % ок
% \qhf  {36}{30,2}{31}         % ок
% \qhf  {36}{30,2}{38}         % Хаффман неоптимален и даже хуже фиксированного
% \qhf  {36}{30,3}{31}         % ок
% \qhfsf{36}{31,0}{32}{32}     % ок
% \qhf  {36}{31,4}{32}         % ок
% \qsf  {36}{31,5}     {32}    % ок
% \qhfsf{36}{31,5}{32}{32}     % ок
% \qhf  {36}{31,6}{32}         % ок
% \qsf  {36}{32,2}     {34}    % ок
% \qhf  {36}{32,7}{32}         % нарушение 1тШ
% \qsf  {36}{33,0}     {34}    % ок
% \qhf  {36}{34,3}{35}         % ок
% \qsf  {36}{34,2}     {35}    % ок
% \qsf  {36}{35,2}     {36}    % ок



% фикс, Iбп, Xф, ШФ
\qsf  {36}{23,47}     {26,964} % длина ШФ — нецелая
\qhfsf{36}{24,6}{27} {26}    % Хаффман хуже ШФ
\qhf  {36}{25,5}{26}         % ок
% \qsf  {36}{25,7}     {27}    % ок
\qsf  {36}{26,2}     {28}    % ок
\qsf  {36}{26,2}     {26}    % 1тШ
\qsf  {36}{26,7}     {27}    % ок
% \qsf  {36}{26,8}     {27}    % ок
\qhfsf{36}{26,8}{25}{27}     % 1тШ
\qhfsf{36}{27,5}{28}{28}     % ок
\qhf  {36}{28,0}{40}         % Хаффман неоптимален и даже хуже фиксированного
% \qsf  {36}{28,2}     {29}    % ок
% \qhf  {36}{28,4}{29}         % ок
\qhf  {36}{20,4}{29}         % Хаффман явно неоптимален
\qhfsf{36}{28,4}{29}{30}     % ок
\qhf  {36}{28,7}{30}         % ок
% \qsf  {36}{29,0}     {30}    % ок
\qsf  {48}{29,0}     {30}    % ок
\qsf  {36}{29,1}     {30}    % ок
\qsf  {36}{2,91}     {30}    % 1тш
\qsf  {36}{29,1}     {29}    % 1тШ
\qsf  {36}{11,1}     {29}    % Хаффман явно неоптимален
\qsf  {36}{29,2}     {30}    % ок
% \qhf  {36}{29,5}{30}         % ок
\qhf  {36}{30,2}{31}         % ок
\qhf  {36}{30,2}{38}         % Хаффман неоптимален и даже хуже фиксированного
% \qhf  {36}{30,3}{31}         % ок
\qhfsf{36}{31,0}{32}{32}     % ок
\qhf  {36}{31,4}{32}         % ок
\qhf  {30}{31,4}{32}         % 1тШ
\qsf  {36}{31,5}     {32}    % ок
% \qhfsf{36}{31,5}{32}{32}     % ок
\qhfsf{36}{3,15}{32}{32}     % 1тш
\qhfsf{48}{31,5}{32}{31}     % неопт, Хаффман хуже ШФ
\qhf  {36}{31,6}{32}         % ок
\qsf  {36}{32,2}     {34}    % ок
% \qsf  {36}{32,2}     {34,6}    % нецелая
\qsf  {36}{2,2}      {34}    % неопт
\qsf  {120}{32,2}    {34}    % ок
\qhf  {36}{32,7}{32}         % нарушение 1тШ
\qsf  {36}{33,0}     {34}    % ок
\qhf  {36}{34,3}{35}         % ок
% \qsf  {36}{34,2}     {35}    % ок
\qsf  {36}{36,2}     {35}    % 1тш
\qsf  {36}{35,2}     {36}    % ок
% 28 ок, 17 неок → убираем часть ок → 28-13=15 ок;  
% -1 неок, -2 ок

\bpLZ {48}{16}{12}{RLE-н}    % ок
\bpLZ {48}{16}{96}{RLE-н}    % ок

\bpLZ {36}{30,2}{108}{RLE-н}    % RLE-н в наихудшем случае увеличивает вдвое, а не втрое
\bpLZ {36}{30,2}{1}{RLE-н}    % RLE-н не м.б таким
% \bpLZ {36}{30,2}{72}{RLE-н}    % ок
\bpLZ {36}{30,2}{30}{RLE-н}    % ок

\bpLZ {48}{47,2}{44}{LZ77-p}    % ок

\mmLZ {48}{44,2}{44}{LZ77-p}    % ок
\mmLZ {48}{44,2}{144}{LZ77-p}    % LZ77-p втрое не увеличивает

\mmLZ {48}{44,2}{144}{RLE-p}    % RLE-p втрое не увеличивает


% \bpLZ {48}{16}{12}{RLE-н}    % ок         есть выше
% % \bpLZ {48}{16}{12}{RLE-н}    % ок
% % \bpLZ {48}{16}{12}{RLE-н}    % ок

\qbmdl{48}{М1}{3}{Хф}{0}   % ок
% \qbmdl{48}{М1}{3}{Хф}{0}   % ок
% \qbmdl{48}{М1}{3}{Хф}{0}   % ок


% \qhf  {30}{31,4}{32}         % 1тШ    есть выше


\end{document}
