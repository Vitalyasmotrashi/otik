%%%%%%%%%%%%%%%%%%%%%%%%%%%%%%%%%%%%%%%%%%%%%%%%%%%%%%%%%%%%%%%%%%%%%%%%%%%%%%%
% Список вопросов и билеты по 2 вопроса
% расстояния подобраны методом тыка
%%%%%%%%%%%%%%%%%%%%%%%%%%%%%%%%%%%%%%%%%%%%%%%%%%%%%%%%%%%%%%%%%%%%%%%%%%%%%%%
\documentclass[a4paper,14pt]{extarticle}
\usepackage[
  a4paper, mag=1000, %nohead,includefoot,
  margin=0mm, %headsep=20mm, footskip=0mm
  top=5mm, bottom=12mm
]{geometry}

\usepackage[T2A]{fontenc}
% \usepackage[utf8x]{inputenc} % некрасивое оглавление: вместо заголовка просит \PrerenderUnicode; а с \PrerenderUnicode — не собирается
\usepackage[utf8]{inputenc} % https://tex.stackexchange.com/questions/48197/error-please-insert-prerenderunicode


\usepackage[english,russian]{babel}
% \usepackage{cyrtimes}	% Times, Arial и Courier одновременно
\usepackage{graphicx}
\usepackage{enumitem}	% Настраиваемые списки
\newcommand\No{№}
\usepackage{mathtools}  % amsmath with extensions
\usepackage{amssymb}  % (otherwise \mathbb does nothing)
\DeclareMathOperator{\sal}{sal}
\DeclareMathOperator{\sar}{sar}
\DeclareMathOperator{\shl}{shl}
\DeclareMathOperator{\shr}{shr}
\newcommand{\hex}[1]{\ensuremath{\mathtt{#1}}}

% \usepackage{forloop}	% Цикл с параметром
\usepackage{tikz} % цикл с параметром — оттула


\usepackage[pdftex,unicode=true,colorlinks]{hyperref}
    \hypersetup{citecolor=blue,linkcolor=blue,urlcolor=blue}

% =============================================================================
\makeatletter
% %     ^
% % ** /!\ Название курса ************************************************* /!\
% %   ----
% не только в билете, но и в списках вопросов и задач

\global\@namedef{courseheading}{Основы теории информации и~кодирования}

\makeatother

\newcounter{idx_ticket}		% счётчик билетов
\setcounter{idx_ticket}{1}


% Один билет =========================
\newcommand\printticket{
\begin{minipage}[c][0.48\textheight]{0.85\linewidth}
\setlength{\parskip}{0.mm}
\setlength{\parindent}{0cm}
\begin{center}
% \vfill
\textbf{
% МОСКОВСКИЙ ИНСТИТУТ ЭЛЕКТРОННОЙ ТЕХНИКИ
Национальный исследовательский университет «МИЭТ»
}
% \vspace{-0.8cm}
\vspace{-0.5\baselineskip}

\noindent\hrulefill

\textbf{\largeЭКЗАМЕНАЦИОННЫЙ БИЛЕТ \No\arabic{idx_ticket}}\addtocounter{idx_ticket}{1}
\end{center}
% \vspace{-0.6cm}

по курсу <<\courseheading>>
\vfill

% \setlist{nolistsep}
% % \begin{enumerate}[noitemsep, 
% % % wide
% % leftmargin = 0em, itemindent=2em]
% \setlist{leftmargin=*}
% \setlist[1]{labelindent=\parindent}
\setlist{nolistsep,noitemsep, leftmargin=*,labelindent=\parindent}
\begin{enumerate}  
        \item \abPrint[Q]{idxQA}.
\vfill  \item \abPrint[Q]{idxQB}.
\end{enumerate}
% \vfill \textbf{Задачи:} дано \printC{idxC}. При построении кодов указывайте все выбираемые вами опции.
% \begin{enumerate}[resume]
\vfill \textbf{Задача:} дано \abPrint[C]{idxC}. При построении кодов \ref{item_task_A}--\ref{item_task_B} указывайте все выбираемые вами опции.
\begin{enumerate}
\vfill\item\label{item_task_A} \abPrint[T]{idxTA}.
\vfill\item\label{item_task_B} \abPrint[T]{idxTB}.
\vfill\item Рассчитайте $I_{\text{РВ}}(C)$, $I_{\text{БП}}(C)$ и~$I_{\text{М1}}(C)$.
Как $I_{X}(C)$ для указанных моделей $X$ должно соотноситься с~длинами кодов \ref{item_task_A}--\ref{item_task_B}? Выполняются ли эти соотношения?
\end{enumerate}
\vfill
% При построении кода необходимо указать все выбранные опции.
% : сортировку по~умолчанию, маркировку ветвей, разрядность и~смещение для $L$ и~для $S$ и~т.\,п.

% \vspace{-\baselineskip}
% \vfill

\vspace{-0.5\baselineskip}
\noindent\hrulefill
% \vspace{-1cm}
\vfill
{%
\footnotesize%
%Билет рассмотрен и~утверждён на заседании института СПИНТех 25.05.22
% Билет рассмотрен и утвержден на заседании УС Института СПИНТЕХ 21.12.2022
% Билет рассмотрен и~утверждён на заседании УС института СПИНТех 07.06.2023 %среда, весна (Д)
% Билет рассмотрен и~утверждён на заседании УС института СПИНТех 18.12.2023 
Билет рассмотрен и~утверждён на заседании УС института СПИНТех 9.12.2024 %(понедельник!!! ???)
\par
\vspace*{1.5\baselineskip}Директор СПИНТех\,\hrulefill\,Гагарина Л.\,Г.
% Директор СПИНТех\,\rlap{\includegraphics[width=48mm]{tmp/gagar_facsimile}}\hrulefill\,Гагарина Л.\,Г.

}
\end{minipage}\centering

}
% =============================================================================


% Печать текста вопроса по счётчику-№ вопроса/задачи (1, 2, 3,... count_XXX)


% Вопрос в списке вопросов и задача в списке задач ====================================================

\newcounter{ab_idx_current}	

\makeatletter
\newcommand\abPrint[2][Q]{\expandafter\@nameuse{#1\arabic{#2}}}
\newcommand\abCreateQuiet[2]{\global\expandafter\@namedef{#1\arabic{ab_idx_current}}{#2}\stepcounter{ab_idx_current}}
\newcommand\abCreateEnumerateItem[2]{\item #2.\abCreateQuiet{#1}{#2}}
\makeatother

% \let\abCreateTagVal\abCreateQuiet



\newcommand\abItem[1]{}

\newcommand\abSeparator{}
\newcommand\abSeparatorWithParams[2]{%      1—maxA, 2—minB
\setcounter{#1}{\value{ab_idx_current}-1}%
\setcounter{#2}{\value{ab_idx_current}}%
}
\newcommand\abCloseList{}



% 1—тег,    2—minA, 3—maxA, 4—minB, 5—maxB  6—заголовок

\newenvironment{abListQuiet}[5][Q]
{%
    \begingroup
    \renewcommand\abItem[1]{\abCreateQuiet{#1}{##1}}
    \setcounter{ab_idx_current}{1}	
    \setcounter{#2}{1}
    \renewcommand\abSeparator{\abSeparatorWithParams{#3}{#4}}
    \renewcommand\abCloseList{\setcounter{#5}{\value{ab_idx_current}-1}}
}
{%    
    \abCloseList%
    \endgroup%
}


\newenvironment{abListEnumerate}[6][Q]
{%
    \begin{abListQuiet}[#1]{#2}{#3}{#4}{#5}
    \renewcommand\abItem[1]{\abCreateEnumerateItem{#1}{##1}}
    \textbf{#6}
    \begin{enumerate}
}
{%    
    \end{enumerate}
    \end{abListQuiet}
}

% =============================================================================




\newcounter{idxQA}	% счётчик для 1-го вопроса билета
\newcounter{idxQB}	% счётчик для 2-го вопроса билета
\newcounter{idxTA}	% счётчик для 1-й задачи билета
\newcounter{idxTB}	% счётчик для 2-й задачи билета
\newcounter{idxC}	

% \newcounter{dQA}	% приращение для 1-го вопроса билета
% \newcounter{dQB}	% приращение для 2-го вопроса билета
% \newcounter{dTA}	% приращение для 1-й задачи билета
% \newcounter{dTB}	% приращение для 2-й задачи билета
% \newcounter{dC}	% приращение 

\newcounter{minQA}	% минимум 1-го вопроса билета в списке вопросов
\newcounter{minQB}	% минимум 2-го вопроса билета в списке вопросов
\newcounter{minTA}	% минимум 1-й задачи билета в списке задач
\newcounter{minTB}	% минимум 2-й задачи билета в списке задач
\newcounter{minC}	

\newcounter{maxQA}	% максимум 1-го вопроса билета   в списке вопросов
\newcounter{maxQB}	% максимум 2-го вопроса билета   в списке вопросов
\newcounter{maxTA}	% максимум 1-й задачи билета     в списке задач
\newcounter{maxTB}	% максимум 2-й задачи билета     в списке задач
\newcounter{maxC}


\newcommand\startticketlist[4]{%
    \setcounter{idxQA}{\value{minQA}}
    \setcounter{idxQB}{\value{minQB}}
    \setcounter{idxTA}{\value{minTA}}
    \setcounter{idxTB}{\value{minTB}}
    \setcounter{idxC}{\value{minC}}
%     \setcounter{dQA}{#1}
%     \setcounter{dQB}{#2}
%     \setcounter{dTA}{#3}
%     \setcounter{dTB}{#4}
%     \setcounter{dC}{1}
}
% Один билет ==================================================================
\newcommand\printticketAndGoNext{%
    \printticket%
%     \addtocounterInRange{idxQA}{dQA}{minQA}{maxQA}
%     \addtocounterInRange{idxQB}{dQB}{minQB}{maxQB}
%     \addtocounterInRange{idxTA}{dTA}{minTA}{maxTA}
%     \addtocounterInRange{idxTB}{dTB}{minTB}{maxTB}
%     \addtocounterInRange{idxC}{dC}{minC}{maxC}
    \stepcounterInRange{idxQA}{minQA}{maxQA}
    \stepcounterInRange{idxQB}{minQB}{maxQB}
    \stepcounterInRange{idxTA}{minTA}{maxTA}
    \stepcounterInRange{idxTB}{minTB}{maxTB}
    \stepcounterInRange{idxC}{minC}{maxC}

}
\newcommand\printIdxAndGoNext{%
    \arabic{idxQA}/%
    \arabic{idxQB}/%
    \arabic{idxTA}/%
    \arabic{idxTB} %
%     \addtocounterInRange{idxQA}{dQA}{minQA}{maxQA}
%     \addtocounterInRange{idxQB}{dQB}{minQB}{maxQB}
%     \addtocounterInRange{idxTA}{dTA}{minTA}{maxTA}
%     \addtocounterInRange{idxTB}{dTB}{minTB}{maxTB}
%     \addtocounterInRange{idxC}{dC}{minC}{maxC}
    \stepcounterInRange{idxQA}{minQA}{maxQA}
    \stepcounterInRange{idxQB}{minQB}{maxQB}
    \stepcounterInRange{idxTA}{minTA}{maxTA}
    \stepcounterInRange{idxTB}{minTB}{maxTB}
    \stepcounterInRange{idxC}{minC}{maxC}
}
% =============================================================================

% Добавление значения счётчика #2 к счётчику #1 с учётом минимального значения счётчика #3 и максимального значения счётчика #4
% \newcommand\addtocounterInRange[4]{
% \addtocounter{#1}{\value{#2}}
% \ifnum \value{#1}>\value{#4}
%     \setcounter{#1}{\value{#3}} % если перевалили за максимум — в минимум
% \fi
% }

\newcommand\stepcounterInRange[3]{
\stepcounter{#1}
\ifnum \value{#1}>\value{#3}
    \setcounter{#1}{\value{#2}} % если перевалили за максимум — в минимум
\fi
}




\usepackage[small,indentafter]{titlesec}
\titlespacing*{\section}{0ex}{3.5ex plus 1ex minus .2ex}{2.3ex plus 0ex}
\titleformat{\section}{\filright\bfseries}{\thesection.}{.5em}{}

\usepackage{titletoc}
\titlecontents{section}[1.5cm]{}{\thecontentslabel.~}{}{\titlerule*[9pt]{.}{\contentspage}}
%%%%%%%%%%%%%%%%%%%%%%%%%%%%%%%%%%%%%%%%%%%%%%%%%%%%%%%%%%%%%%%%%%%%%%%%%%%%%%%
\begin{document}


\newgeometry{margin=20mm}

\clearpage

\phantomsection\addcontentsline{toc}{section}{Вопросы и задачи}
\begin{abListEnumerate}[Q]{minQA}{maxQA}{minQB}{maxQB}{Контрольные (экзаменационные) вопросы по~курсу <<\courseheading>>}
\abItem{Вероятностный подход к измерению информации}
\abItem{Энтропия источника данных, первая теорема Шеннона для сжатия}
\abItem{Энтропийное кодирование, модель источника}
\abItem{Метод энтропийного сжатия Шеннона"--~Фано}
\abItem{Метод энтропийного сжатия Хаффмана}
% \abItem{Арифметическое кодирование}

% окончание первых (в билете) вопросов — ab_idx_current после \abItem{} — это уже номер следующего
\abSeparator

\abItem{Кодирование длин повторений (RLE): идея, основные опции. «Наивный» RLE. RLE с~флаг-битом сжатая/несжатая цепочка}
\abItem{Кодирование длин повторений (RLE): идея, основные опции. «Наивный» RLE. RLE с~односимвольным префиксом сжатой цепочки в несжатом тексте}


\abItem{Метод словарного сжатия LZ77: идея, основные опции. Концепт Зива---Лемпеля от 1977 года}
\abItem{Метод словарного сжатия LZ77: идея, основные опции. LZ77 с~флаг-битом ссылка/символ}
\abItem{Метод словарного сжатия LZ77: идея, основные опции. LZ77 с~односимвольным префиксом ссылки в несжатом тексте}




% \abItem{Семейство алгоритмов словарного сжатия LZ78. Концепт Зива---Лемпеля от 1978 года}
% \abItem{Семейство алгоритмов словарного сжатия LZ78. LZW}


% % \abItem{Сжатие с потерями}
% % \abItem{Дискретное преобразование Фурье (ДПФ)}
% % \abItem{Быстрое ДПФ (БПФ)}
% % \abItem{Дискретное косинусное преобразование}
% % \abItem{Преобразование Хаара}
% \abItem{Ёмкость канала передачи данных, первая теорема Шеннона для канала}
% % \abItem{Условная энтропия и~относительная информация двух источников данных, вторая теорема Шеннона}
% % \abItem{Двоичный симметричный канал}
% \abItem{Простейшие помехоустойчивые коды}
% \abItem{Код Хэмминга}
% % % \abItem{Алгебры, группоиды, полугруппы, моноиды}
% % % \abItem{Группы}
% % % \abItem{Циклические группы и~примитивные элементы}
% % % \abItem{Кольца, тела}
% % % \abItem{Поля}
% % \abItem{Кольца, тела, поля}
% % \abItem{Кольцо многочленов над полем}
% % % \abItem{Делимость многочленов}
% % % \abItem{Неприводимые многочлены}
% % \abItem{Делимость многочленов. Неприводимые многочлены}
% % \abItem{Конечные поля}
% % \abItem{Полиномиальные коды}
% % % \abItem{Циклические полиномиальные коды}
% % % \abItem{Циклические коды Рида"--~Соломона}
% % % \abItem{Двоично-десятичное кодирование (ДДК)}
% % % \abItem{Свойства двоично-десятичных кодов}
% % % \abItem{Коды Эмери}
% % \abItem{Натуральный двоичный код. Двоично-десятичное кодирование (ДДК), виды, примеры}
% % \abItem{Код Грея}
% % % \abItem{Единичный код}
% % % \abItem{Код Джонсона}
% % \abItem{Единичный код, код Джонсона}

\abItem{Структура текстовых файлов. ASCII. Расширения ASCII, кодировки русского языка}
\abItem{Структура текстовых файлов. Unicode. UTF-8, UTF-16, UTF-32}
% \abItem{}
% \abItem{}
% \abItem{}
% \abItem{}
% \abItem{}
% \abItem{}

\end{abListEnumerate}





\clearpage

\newcommand\Ctri[1]{\abItem{сообщение $C=\hex{#1}$ (в~байте \mbox{$k=3$~бита)}}}

\begin{abListQuiet}[C]{minC}{}{}{maxC}
% \Ctetra{0000 \, 0000 \, 0123 \, 4567 ~~ 89AB \, CDEF \, 9ABC \, DEF9}
% \Ctetra{0000 \, 1111 \, 0123 \, 4567 ~~ 89AB \, CDEF \, DEFD \, EFDE}
% \Ctetra{0123 \, 4567 \, 89AB \, CDEF ~~ FEDC \, CDEF \, FEDC \, CDEF}
% \Ctetra{0123 \, 4567 \, 89AB \, CDEF ~~ 0000 \, 1111 \, 2222 \, 2222}
% \Ctetra{0123 \, 4567 \, 89AB \, CDEF ~~ 4567 \, 89AB \, CDEF \, 2222}
% \Ctetra{0123 \, 4567 \, 89AB \, CDEF ~~ 0123 \, 4567 \, 89AB \, CDEF}
% \Ctetra{0011 \, 2233 \, 4455 \, 4455 ~~ 4455 \, 5555 \, 6789 \, CDEF}

\Ctri{0000 \, 0000 \, 0123 \, 4567 ~~ 7777 \, 7777 \, 2223 \, 3322}
\Ctri{0000 \, 0000 \, 0123 \, 3333 ~~ 3434 \, 3434 \, 0001 \, 4567}
\Ctri{0001 \, 2222 \, 2223 \, 3333 ~~ 4555 \, 6677 \, 6776 \, 7767}
\Ctri{1111 \, 1111 \, 0123 \, 4567 ~~ 1231 \, 2312 \, 1212 \, 1212}
\Ctri{0000 \, 2444 \, 0123 \, 3333 ~~ 4567 \, 7771 \, 7772 \, 7727}


\end{abListQuiet}






\newcommand\taskbp[1]{\abItem{Сожмите $C$ кодом из семейства #1 без учёта контекста}}
\newcommand\taskma[1]{\abItem{Сожмите $C$ кодом из семейства #1}}


\begin{abListQuiet}[T]{minTA}{maxTA}{minTB}{maxTB}%{Контрольные (экзаменационные) задачи по~курсу <<\courseheading>>}



% \task{Сожмите сообщение $C$ алгоритмом Хаффмана. 
% Рассчитайте $I_{\text{БП}}(C)$
% }
% \task{Сожмите сообщение $C$ алгоритмом Шеннона-Фано}
% \task{Сожмите сообщение $C$ алгоритмом }

\taskbp{Хаффмана}
\taskbp{Шеннона-Фано}
% \taskbp{Шеннона}

\taskma{Хаффмана с~учётом одного предшествующего символа}

% окончание первых (в билете) задач
\abSeparator

% \task{Сожмите сообщение $C$ алгоритмом RLE с~флаг-битом сжатая/несжатая цепочка}
% \task{Сожмите сообщение $C$ алгоритмом RLE с~односимвольным префиксом сжатой цепочки в несжатом тексте}
% 
% % \task{Сожмите сообщение $C$ алгоритмом LZ77-концептом 1977 г.}
% \task{Сожмите сообщение $C$ алгоритмом LZ77 с~флаг-битом ссылка/символ}
% \task{Сожмите сообщение $C$ алгоритмом LZ77 с~односимвольным префиксом ссылки в несжатом тексте}
% 
% % \task{Сожмите сообщение $C$ алгоритмом LZ78-концептом 1978 г.}
% % \task{Сожмите сообщение $C$ алгоритмом LZW}

\taskma{RLE с~флаг-битом сжатая/несжатая цепочка}
\taskma{RLE с~односимвольным префиксом сжатой цепочки в~несжатом тексте}

\taskma{LZ77 с~флаг-битом ссылка/символ (флаг-биты группируются по~$k$ штук во~флаг-байты)}
\taskma{LZ77 с~односимвольным префиксом ссылки в~несжатом тексте}

\end{abListQuiet}

% 2024—2025: 3(?) первых задачи, 4(?) вторых

% \end{minipage}
\restoregeometry
\pagestyle{empty}	% Страницы без нумерации










\setlength{\leftmargini}{2cm}

\phantomsection\addcontentsline{toc}{section}{Экзаменационные билеты}

\startticketlist{1}{1}{1}{1}

\foreach \ticketpage in {1,...,15}{
    \printticketAndGoNext\vfill\printticketAndGoNext
%     \printIdxAndGoNext\par\printIdxAndGoNext\par
}




\end{document}
